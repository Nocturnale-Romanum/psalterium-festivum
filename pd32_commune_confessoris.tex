% !TEX TS-program = lualatex
% !TEX encoding = UTF-8

\documentclass[psalterium-festivum.tex]{subfiles}

\ifcsname preamble@file\endcsname
  \setcounter{page}{\getpagerefnumber{M-pd32_commune_confessoris}}
\fi

\begin{document}

\feast{CONP}{Commune Confessoris}
	{Commune Sanctorum}{Commune Confessoris}{2}{}{}{}{}{}{}
\thispagestyle{empty}
\addcontentsline{toc}{section}{Commune Confessoris}
\rubric{Extra Tempus Paschale:}
\gscore{COPOIc}{I}{}{Regem Confessorum\idxnewline(tonus solemnis)}
\rubric{Tempore Paschali:}
\gscore{COPOId}{I}{}{Regem Confessorum\idxnewline(tempore paschali)}
\gscore{COPOF1H}{H}{}{Iste Confessor}
\nocturn{1}
\gscore{COPON1A1}{A}{1}{Beatus vir... prosperabuntur}
\tptresrubric
\psalmus{1}{3}
\gscore{COPON1A2}{A}{2}{Beatus iste Sanctus}
\psalmus{2}{1}
\gscore{COPON1A3}{A}{3}{Tu es gloria mea}
\psalmus{3}{8}
\versiculustpall{Amávit eum Dóminus, et ornávit eum.}{Stolam glóriæ índuit eum.}
\nocturn{2}
\gscore{COPON2A1}{A}{4}{Invocantem exaudivit}
\tptresrubric
\psalmus{4}{2}
\gscore{COPON2A2}{A}{5}{Laetentur omnes}
\psalmus[label]{5}{8}
\gscore{COPON2A3}{A}{6}{Domine Dominus... qui gloria\newline\null}
\psalmus{8}{1}
\rubric{Pro Confessore Pontifice}
\versiculustpall{Elégit eum Dóminus sacerdótem sibi.}{Ad sacrificándum ei hóstiam laudis.}
\rubric{Pro Confessore non Pontifice}
\versiculustpall{Os justi meditábitur sapiéntiam.}{Et lingua ejus loquétur justítiam.}
\nocturn{3}
\gscore{COPON3A1}{A}{7}{Domine iste Sanctus}
\tptresrubric
\psalmus{14}{8}
\gscore{COPON3A2}{A}{8}{Vitam petiit}
\psalmus{20}{8}
\gscore{COPON3A3}{A}{9}{Hic accipiet}
\psalmus{23}{7}
\pagebreak
\rubric{Pro Confessore Pontifice}
\versiculustpall{Tu es sacérdos in ætérnum.}{Secúndum órdinem Melchísedech.}
\rubric{Pro Confessore non Pontifice}
\versiculustpall{Lex Dei ejus in corde ipsíus.}{Et non supplantabúntur gressus ejus.}


\feast{MU}{Commune Virginum aut non Virginum}
	{Commune Sanctorum}{Commune Virginum aut non Virginum}{2}{}{}{}{}{}{}
\addcontentsline{toc}{section}{Commune Virginum aut non Virginum}
\rubric{Pro Virgine, extra Tempus Paschale:}
\gscore{MUVXIc}{I}{}{Regem Virginum\idxnewline(tonus solemnis)}
\rubric{Pro Virgine, Tempore Paschali:}
\gscore{MUVXId}{I}{}{Regem Virginum\idxnewline(tempore paschali)}
\rubric{Pro una non Virgine:}
\gscore{MUNXIa}{I}{}{Laudemus\idxnewline(unius non Virginis)}
\pagebreak
\rubric{Pro plurimis non Virginis:}
\gscore{MUNXIb}{I}{}{Laudemus (plurimarum\idxnewline{}non Virginum)}
\rubric{Pro Virgine non Martyre, strophæ 1, 4 et 5 tantum dicuntur.\\Pro non Virgine, strophæ 4 et 5 tantum dicuntur.}
\gscore{MUVMHa}{H}{}{Virginis proles}

\nocturn{1}
\gscore{MUXXN1A1}{A}{1}{O quam pulchra}
\tptresrubric
\pagebreak
\psalmus[label]{8}{7}
\rubric{Pro Virgine:}
\gscore{MUVXN1A2}{A}{2}{Ante torum}
\rubric{Pro non Virgine:}
\gscore{MUNXN1A2}{A}{2}{Laeva ejus}
\pagebreak
\psalmus{18}{4e}
\gscore{MUXXN1A3}{A}{3}{Revertere revertere}
\psalmus{23}{7}
\versiculustpall{Spécie tua et pulchritúdine tua.}{Inténde, próspere procéde, et regna.}

\nocturn{2}
\gscore{MUXXN2A1}{A}{4}{Specie tua}
\tptresrubric
\pagebreak
\psalmus{44}{7}
\gscore{MUXXN2A2}{A}{5}{Adjuvabit eam}
\psalmus{45}{7}
\gscore{MUXXN2A3}{A}{6}{Aquae multae}
\pagebreak\null\vspace{2\baselineskip}
\psalmus{47}{8}
\versiculustpall{Adjuvábit eam Deus vultu suo.}{Deus in médio ejus, non commovébitur.}
\vspace{3\baselineskip}\null\pagebreak\null\vspace{\baselineskip}
\nocturn{3}
\gscore{MUXXN3A1}{A}{7}{Nigra sum}
\tptresrubric
\psalmus{95}{3}
\gscore{MUXXN3A2}{A}{8}{Trahe me}
\psalmus{96}{3}
\gscore{MUXXN3A3}{A}{9}{Veni sponsa Christi}
\psalmus{97}{8}
\versiculustpall{Elégit eam Deus, et præelégit eam.}{In tabernáculo suo habitáre facit eam.}

\end{document}