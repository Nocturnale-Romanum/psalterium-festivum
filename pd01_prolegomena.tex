% !TEX TS-program = lualatex
% !TEX encoding = UTF-8

\documentclass[psalterium-dominicis.tex]{subfiles}

\ifcsname preamble@file\endcsname
  \setcounter{page}{\getpagerefnumber{M-pd01_prolegomena}}
\fi

\begin{document}

\begin{titlepage}
\begin{center}
\null\vspace{1cm}
{\Large\sc Nocturnale Romanum}

\vspace{5mm}

{\large\sc Tomus I\textsc{a}}

\vspace{4cm}

{\Huge\sc Psalterium Romanum}

\vspace{1cm}

{\Large\sc pro nocturnis horis\\in dominicis et festis majoribus}

\vspace{2cm}

{\large\sc instrumentum laboris}

\vfill

MMXXIII

\end{center}
\end{titlepage}
\addcontentsline{toc}{section}{Prolegomena}

\feast{OR}{Prolegomena}
	{Prolegomena}{Prolegomena}{2}{}{}{}{}{}{}
\thispagestyle{empty}
	
\intermediatetitle{The Once and Future Roman Night Office}

{\setlength{\parindent}{5mm}\small

This book is the first attempt at publishing excerpts of what will, God willing, become a complete edition of the \emph{Nocturnale Romanum}.

It aims at bringing something of the past into the future, and at putting it to good use in the present; in this sense can it be said to be, to the highest degree, traditional.

Its melodies have been critically established using dozens of the most ancient manuscripts; in a few places, as is the case also in the editions of the \emph{Antiphonale Romanum},
the text of the antiphons differs slightly from that found in the Breviary, heeding the overwhelming consensus of the sources, where such a consensus exists. 
Antiphons of very recent introduction have been given melodies of the existing tradition of Gregorian Chant.

This book looks up to the horizon of a future authentic restoration of the Roman liturgy; this is, in part, why it uses the text of hymns as before the reforms of pope Urban VIII.

It is the book proper to the ordinary choir member: it contains everything that an ordinary choir member will sing at Matins, namely, the Invitatory (without its psalm, sung by a cantor), 
the Hymn, Antiphons, Psalms, and Versicles, as well as the common tones of those parts sung by all.

This book attempts to enable the Church at large to render unto God a fitting praise in the present; this is, in part, why it uses rhythmic signs, based on the rules of the \emph{mora vocis}
and on the rhythmic indications of the adiastematic manuscripts, where they exist, 
because the faithful are accustomed to these signs more than to reproductions of ancient neumes: a table of equivalence is found below. 
This is also why this book is laid out so that it may be used by those celebrating Divine Office according to either 1960 or \emph{Divino Afflatu} rubrics.

The selection of feasts found in this book corresponds more or less to those of the first and second class in the 1960 system, 
and to those of \emph{duplex majus} rank and above in the older system;
but not all \emph{duplex majus} that were downgraded to the third class in 1960 are included.

It is assumed that all users of this book have an \emph{Ordo} prepared for them or by them, and need no indications of which parts to use or omit depending on the rules that apply to them;
for instance, someone celebrating according th 1960 rubrics would know to use the ferial psalmody, instead of that provided in the present book, on the Decollation of St. John the Baptist,
it having been downgraded to a third class feast. Therefore, the rubrics given here are minimal.

This book is also, and most importantly at present, a \textbf{draft}. We, the editing team, are entirely too few to bring it up to proper standards by ourselves. 
Please give your feedback on\\{\footnotesize\url{https://github.com/Nocturnale-Romanum/nocturnale-romanum/issues}}

}

\intermediatetitle{L'Office nocturne romain, passé et à venir}

{\setlength{\parindent}{5mm}\small

Ce livre est la première tentative d'imprimer de larges extraits de ce qui deviendra, si Dieu le veut, une édition complète du \emph{Nocturnale Romanum}.

Il vise à conserver quelque chose du passé et à le transmettre pour l'avenir, tout en en faisant bon usage dans le présent; ce en quoi il peut être qualifié, au plus haut point, de «traditionnel». 

Ses mélodies ont été établies critiquement d'après des dizaines des plus anciens manuscrits; à quelques endroits, comme c'est le cas dans les éditions de l'\emph{Antiphonale Romanum},
le texte des antiennes diffère légèrement de celui du Bréviaire, par respect pour le consensus écrasant des sources, là où ce consensus existe.
Les antiennes d'introduction très récente ont reçu des mélodies tirées de la tradition grégorienne existante.

Ce livre lève les yeux vers l'horizon d'une future restauration authentique de la liturgie romaine; c'est l'une des raisons pour lesquelles il emploie pour les hymnes le texte antérieur à la réforme du pape Urbain VIII.

Il s'agit du livre propre à celui qui assiste à l'office sans y tenir de rôle particulier: il contient toutes les parties que celui-ci devra chanter aux Matines, c'est à dire l'invitatoire (sans son psaume, chanté par un chantre), l'hymne, les antiennes, psaumes et versicules, ainsi que les tons communs des parties ordinaires chantées par tous.

Ce livre veut permettre au plus grand nombre de fidèles de rendre à Dieu dès aujourd'hui un culte digne de lui; c'est pourquoi il emploie les signes rythmiques, 
selon les règles de la \emph{mora vocis} et d'après le consensus rythmique des manuscrits adiastématiques, puisque les fidèles sont plus habitués à ces signes qu'à la copie des neumes anciens:
une table de correspondance est donnée ci-après. C'est aussi la raison pour laquelle ce livre, autant que possible, permet la célébration de l'Office Divin selon les rubriques de 1960 ou celles
définies dans \emph{Divino Afflatu}.

La sélection de fêtes présentes dans ce livre correspond plus ou moins à celles de première et deuxième classe dans le système 1960, et aux doubles majeurs (et supérieurs) de l'ancien système; 
mais certains doubles majeurs qui ont été réduits à des fêtes de troisième classe en 1960 sont absents.

On fait l'hypothèse que les usagers de ce livre ont un \emph{Ordo} préparé pour eux ou par eux, et n'ont pas besoin d'indications sur les parties à dire ou à omettre selon les règles
qui s'appliquent à eux; par exemple, si l'on célèbre selon les rubriques de 1960, on saura qu'il faut employer la psalmodie fériale, et non celle donnée ici,
le jour de la Décollation de Saint Jean Baptiste, puisque c'est désormais une fête de troisième classe. C'est pourquoi les rubriques ont été réduites à leur plus simple expression.

Ce livre est aussi, et surtout --- pour le moment --- un \textbf{brouillon}. L'équipe d'édition est beaucoup trop réduite pour en amener la qualité, par elle-même, à un niveau satisfaisant.
Merci de bien vouloir signaler les corrections nécessaires sur\\{\footnotesize\url{https://github.com/Nocturnale-Romanum/nocturnale-romanum/issues}}

}

\intermediatetitle{Tabella neumatum}

{\gresetnabc{1}{visible}
\gresetclef{invisible}
\gresetinitiallines{0}
\gregorioscore{\subfix{nocturnale-romanum/gabc/neumata}}
}

\cleardoublepage


\end{document}