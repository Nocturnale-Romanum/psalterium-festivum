% !TEX TS-program = lualatex
% !TEX encoding = UTF-8

\documentclass[psalterium-dominicis.tex]{subfiles}

\ifcsname preamble@file\endcsname
  \setcounter{page}{\getpagerefnumber{M-nr31_commune_martyrum}}
\fi

\begin{document}
\feast{UMEX}{Commune unius Martyris\\extra Tempus Paschale}
	{Commune Sanctorum}{Commune unius Martyris extra T. P.}{2}{}{}{}{}{}{}
\addcontentsline{toc}{section}{Commune Martyrum}

\gscore{UMEXIc}{I}{}{Regem Martyrum (1.2. classis)}
\gscore{UMEXHa}{H}{}{Deus tuorum militum (tonus prior)}
\gscore{UMEXHb}{H}{}{Deus tuorum militum (tonus alter)}
\nocturn{1}
\gscore{UMEXN1A1}{A}{1}{In lege Domini fuit voluntas}
\psalmus{1}{}
\gscore{UMEXN1A2}{A}{2}{Praedicans praeceptum}
\psalmus{2}{}
\gscore{UMEXN1A3}{A}{3}{Voce mea ad Dominum clamavi et exaudivit}
\psalmus{3}{}
\versiculus{Glória et honóre coronásti eum, Dómine.}{Et constituísti eum super ópera mánuum tuárum.}
\nocturn{2}
\gscore{UMEXN2A1}{A}{4}{Filii hominum scitote}
\psalmus{4}{}
\gscore{UMEXN2A2}{A}{5}{Scuto bonae voluntatis tuae}
\psalmus{5}{}
\gscore{UMEXN2A3}{A}{6}{In universa terra gloria et honore}
\psalmus{8}{}
\versiculus{Posuísti, Dómine, super caput ejus.}{Corónam de lápide pretióso.}
\nocturn{3}
\gscore{UMEXN3A1}{A}{7}{Justus Dominus... aequitatem}
\psalmus{10}{}
\gscore{UMEXN3A2}{A}{8}{Habitabit in tabernaculo tuo}
\psalmus{14}{}
\gscore{UMEXN3A3}{A}{9}{Posuisti Domine super caput}
\psalmus{20}{}
\versiculus{Magna est glória ejus in salutári tuo.}{Glóriam et magnum decórem impónes super eum.}

\feast{PMEX}{Commune plurimorum Martyrum\\extra Tempus Paschale}
	{Commune Sanctorum}{Commune plurimorum Martyrum extra T. P.}{2}{}{}{}{}{}{}
\gscore{UMEXIc}{I}{}{Regem Martyrum (1.2. classis)}
\gscore{PMEXH}{H}{}{Christo profusum sanguinem}
\nocturn{1}
\rubric{In primo Nocturno Psalmi ut in Communi unius Martyris extra Tempus Paschale, ut supra pag.\ \pageref{M-UMEXN1A1}.}
\gscore{PMEXN1A1}{A}{1}{Secus decursus}
\gscore{PMEXN1A2}{A}{2}{Tamquam aurum}
\gscore{PMEXN1A3}{A}{3}{Si coram hominibus}
\versiculus{Lætámini in Dómino et exsultáte, justi.}{Et gloriámini, omnes recti corde.}
\nocturn{2}
\gscore{PMEXN2A1}{A}{4}{Dabo sanctis meis}
\psalmus{14}{}
\gscore{PMEXN2A2}{A}{5}{Sanctis qui in terra}
\psalmus{15}{}
\gscore{PMEXN2A3}{A}{6}{Sancti qui sperant}
\psalmus{23}{}
\versiculus{Exsúltent justi in conspéctu Dei.}{Et delecténtur in lætítia.}
\nocturn{3}
\gscore{PMEXN3A1}{A}{7}{Justi autem in perpetuo}
\psalmus{32}{}
\gscore{PMEXN3A2}{A}{8}{Tradiderunt corpora sua}
\psalmus{33}{}
\gscore{PMEXN3A3}{A}{9}{Ecce merces sanctorum}
\psalmus{45}{}
\versiculus{Justi autem in perpétuum vivent.}{Et apud Dóminum est merces eórum.}

\feast{MRTP}{Commune unius aut plurimorum Martyrum\\Tempore Paschali}
	{Commune Sanctorum}{Commune unius aut plurimorum Martyrum T. P.}{2}{}{}{}{}{}{}
\gscore{MRTPI}{I}{}{Exsultent in Domino Sancti Alleluia}
\rubric{In Festis unius Martyris Hymnus \scorename{UMEXHa}, pag.\ \pageref{M-UMEXHa} aut \pageref{UMEXHb}; et in Festis plurimorum Martyrum Hymnus \scorename{PMEXH}, pag.\ \pageref{M-PMEXH} cantatur.}

\rubric{In singulis nocturnis tres Psalmi sub una Antiphona cantentur. Omnes Antiphonæ, Versiculi et Responsoria, exceptis 5.\ et 6.\ ut infra notantur, summuntur ex Communi Apostolorum et Evangelistorum tempore paschali, pag. \pageref{M-APTPN1A}, sed Psalmi aut e communi unius Martyris aut plurimorum Martyrum juxta Festi peragendi qualitatem, nisi aliter notetur.}

\end{document}