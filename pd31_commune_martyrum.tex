% !TEX TS-program = lualatex
% !TEX encoding = UTF-8

\documentclass[psalterium-dominicis.tex]{subfiles}

\ifcsname preamble@file\endcsname
  \setcounter{page}{\getpagerefnumber{M-pd31_commune_martyrum}}
\fi

\begin{document}
\feast{UMEX}{Commune unius Martyris\\extra Tempus Paschale}
	{Commune Sanctorum}{Commune unius Martyris extra T. P.}{2}{}{}{}{}{}{}
\addcontentsline{toc}{section}{Commune Martyrum}

\gscore{UMEXIc}{I}{}{Regem Martyrum}
\gscore{UMEXHa}{H}{}{Deus tuorum militum}
\nocturn{1}
\gscore{UMEXN1A1}{A}{1}{In lege Domini fuit voluntas}
\psalmus{1}{}
\gscore{UMEXN1A2}{A}{2}{Praedicans praeceptum}
\psalmus{2}{}
\gscore{UMEXN1A3}{A}{3}{Voce mea ad Dominum clamavi et exaudivit}
\psalmus{3}{}
\versiculus{Glória et honóre coronásti eum, Dómine.}{Et constituísti eum super ópera mánuum tuárum.}
\nocturn{2}
\gscore{UMEXN2A1}{A}{4}{Filii hominum scitote}
\psalmus{4}{}
\gscore{UMEXN2A2}{A}{5}{Scuto bonae voluntatis tuae}
\psalmus{5}{}
\gscore{UMEXN2A3}{A}{6}{In universa terra gloria et honore}
\psalmus{8}{}
\versiculus{Posuísti, Dómine, super caput ejus.}{Corónam de lápide pretióso.}
\nocturn{3}
\gscore{UMEXN3A1}{A}{7}{Justus Dominus... aequitatem}
\psalmus{10}{}
\gscore{UMEXN3A2}{A}{8}{Habitabit in tabernaculo tuo}
\psalmus{14}{}
\gscore{UMEXN3A3}{A}{9}{Posuisti Domine super caput}
\psalmus{20}{}
\versiculus{Magna est glória ejus in salutári tuo.}{Glóriam et magnum decórem impónes super eum.}

\end{document}