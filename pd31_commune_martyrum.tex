% !TEX TS-program = lualatex
% !TEX encoding = UTF-8

\documentclass[psalterium-dominicis.tex]{subfiles}

\ifcsname preamble@file\endcsname
  \setcounter{page}{\getpagerefnumber{M-pd31_commune_martyrum}}
\fi

\begin{document}
\feast{UMEX}{Commune unius Martyris\\extra Tempus Paschale}
	{Commune Sanctorum}{Commune unius Martyris extra T. P.}{2}{}{}{}{}{}{}
\thispagestyle{empty}
\addcontentsline{toc}{section}{Commune Martyrum}
\rubric{Tonus solemnis:}
\gscore{UMEXIc}{I}{}{Regem Martyrum!Tonus solemnis}
\rubric{Tonus festivus:}
\gscore{UMEXIb}{I}{}{Regem Martyrum!Tonus festivus}
\gscore{UMEXHa}{H}{}{Deus tuorum militum}
\nocturn{1}
\gscore{UMEXN1A1}{A}{1}{In lege Domini}
\pagebreak
\psalmus{1}{1}
\gscore{UMEXN1A2}{A}{2}{Praedicans praeceptum}
\psalmus{2}{1}
\gscore{UMEXN1A3}{A}{3}{Voce mea}
\psalmus{3}{7}
\versiculus{Glória et honóre coronásti eum, Dómine.}{Et constituísti eum super ópera mánuum tuárum.}
\nocturn{2}
\gscore{UMEXN2A1}{A}{4}{Filii hominum scitote}
\psalmus{4}{1}
\gscore{UMEXN2A2}{A}{5}{Scuto bonae voluntatis tuae}
\psalmus{5}{2}
\vspace{\baselineskip}
\gscore{UMEXN2A3}{A}{6}{In universa terra}
\pagebreak
\psalmus{8}{2}
\versiculus{Posuísti, Dómine, super caput ejus.}{Corónam de lápide pretióso.}
\nocturn{3}
\gscore{UMEXN3A1}{A}{7}{Justus Dominus... aequitatem}
\psalmus{10}{7}
\gscore{UMEXN3A2}{A}{8}{Habitabit in tabernaculo}
\psalmus[label]{14}{4e}
\gscore{UMEXN3A3}{A}{9}{Posuisti Domine}
\psalmus{20}{4e}
\versiculus{Magna est glória ejus in salutári tuo.}{Glóriam et magnum decórem impónes super eum.}

\feast{PMEX}{Commune plurimorum Martyrum\\extra Tempus Paschale}
	{Commune Sanctorum}{Commune plurimorum Martyrum extra T. P.}{2}{}{}{}{}{}{}
\thispagestyle{empty}
\rubric{Invitatorium \normaltext{Regem Martyrum} pag.\ \pageref{M-UMEXIc}.}
\gscore{PMEXH}{H}{}{Aeterna Christi munera (pro Martyris)}

\nocturn{1}
\gscore{PMEXN1A1}{A}{1}{Secus decursus}
\psalmus{1}{4e}
\gscore{PMEXN1A2}{A}{2}{Tamquam aurum}
\psalmus{2}{7}
\gscore{PMEXN1A3}{A}{3}{Si coram hominibus}
\psalmus{3}{7}
\versiculus{Lætámini in Dómino et exsultáte, justi.}{Et gloriámini, omnes recti corde.}

\nocturn{2}
\gscore{PMEXN2A1}{A}{4}{Dabo sanctis meis}
\psalmus{14}{8}
\gscore{PMEXN2A2}{A}{5}{Sanctis qui in terra}
\psalmus{15}{4e}
\vspace{\baselineskip}
\gscore{PMEXN2A3}{A}{6}{Sancti qui sperant}
\psalmus{23}{3}
\versiculus{Exsúltent justi in conspéctu Dei.}{Et delecténtur in lætítia.}

\nocturn{3}
\gscore{PMEXN3A1}{A}{7}{Justi autem}
\psalmus{32}{4e}
\vfill
{\grechangedim{baselineskip}{65pt plus 1pt minus 5pt}{scalable}
	\gscore{PMEXN3A2}{A}{8}{Tradiderunt corpora}
}
\pagebreak
\psalmus[label]{33}{1}
\gscore{PMEXN3A3}{A}{9}{Ecce merces sanctorum}
\psalmus{45}{4e}
\versiculus{Justi autem in perpétuum vivent.}{Et apud Dóminum est merces eórum.}

\feast{MRTP}{Commune unius aut plurimorum Martyrum\\Tempore Paschali}
	{Commune Sanctorum}{Commune unius aut plurimorum Martyrum T. P.}{2}{}{}{}{}{}{}
\thispagestyle{empty}

\gscore{MRTPI}{I}{}{Exsultent in Domino}

\rubric{In Festis unius Martyris Hymnus \scorename{UMEXHa}, pag.\ \pageref{M-UMEXHa}, et in Festis plurimorum Martyrum Hymnus \scorename{PMEXH}, pag.\ \pageref{M-PMEXH}.}

\rubric{In singulis nocturnis tres Psalmi sub una Antiphona dicuntur. Antiphonæ et Versus ex Communi Apostolorum et Evangelistarum Tempore Paschali, pag. \pageref{M-APTPN1A}, sed Psalmi aut e communi unius Martyris aut plurimorum Martyrum pro qualitate Festi.}

\end{document}


