% !TEX TS-program = lualatex
% !TEX encoding = UTF-8

\documentclass[psalterium-dominicis.tex]{subfiles}

\ifcsname preamble@file\endcsname
  \setcounter{page}{\getpagerefnumber{M-pd20_psalterium_ordinarium}}
\fi

\begin{document}
\feast{OR}{Ordinarium Divini Offici\\ ad Matutinum}
	{Ordinarium}{Ordinarium}{1}{}{}{}{}{}{}
\addcontentsline{toc}{chapter}{Ordinarium Divini Officii ad Matutinum}

\intermediatetitle{Ante Divinum Officium}

\begin{multicols}{2}
\lettrine{A}{peri}, Dómine, os meum ad benedicéndum nomen sanctum tuum:
munda quoque cor meum ab ómnibus vanis, pervérsis et aliénis cogitatiónibus; 
intelléctum illúmina, afféctum inflámma, ut digne, atténte ac devóte
hoc officium recitáre váleam, et exaudíri mérear
ante conspéctum divinae Majestátis tuae. Per Christum Dóminum nostrum. Amen.

~

\lettrine{D}{ómine}, in unióne illíus divínæ intentiónis,
qua ipse in terris laudes Deo persolvísti,
has tibi horas \rubric{(vel} hanc tibi horam\rubric{)} persólvo.
\end{multicols}

\sep

\begin{multicols}{2}
\lettrine{P}{ater noster}, qui es in cælis, sanctificétur nomen tuum.
Advéniat regnum tuum. Fiat volúntas tua, sicut in cælo et in terra.
Panem nostrum quotidiánum da nobis hódie.
Et dimítte nobis débita nostra, sicut et nos dimíttimus debitóribus nostris.
Et ne nos indúcas in tentatiónem: sed líbera nos a malo. Amen.

\lettrine{A}{ve María}, grátia plena, Dóminus tecum:
benedícta tu in muliéribus, et benedíctus fructus ventris tui Jesus.
Sancta María, Mater Dei, ora pro nobis peccatóribus,
nunc et in hora mortis nostræ. Amen.

\lettrine{C}{redo in Deum}, Patrem omnipoténtem, Creatórem cæli et terræ.
Et in Jesum Christum, Fílium ejus únicum, Dóminum nostrum,
qui concéptus est de spíritu Sancto, natus ex María Virgine,
passus sub Póntio Piláto, crucifixus, mórtuus et sepúltus:
descéndit ad ínferos: tértia die resurréxit a mórtuis;
ascéndit ad cælos, sedet ad déxteram Patris omnipoténtis:
inde ventúrus est judicáre vivos et mórtuos.
Credo in Spíritum sanctum, sanctam Ecclésiam cathólicam,
Sanctórum communiónem, remissiónem peccatórum,
carnis resurrectiónem, vitam ætérnam. Amen.
\end{multicols}

\pagebreak

\gscore{ORIa}{T}{}{Domine labia mea!Tonus simplex}
\gscore{ORIb}{T}{}{Domine labia mea!Tonus festivus}

\intermediatetitle{Invitatorium et Hymnus}

\rubric{Postea dicitur conveniens Invitatorium, quod ante Psalmum bis cantatur,
et ad singulos ejusdem Psalmi versus
vel integrum vel dimidiatum ab asterisco \normaltext{\GreSpecial{*}} repetitur.}

\label{ORIP2d}
\label{ORIP3c}
\label{ORIP3e}
\label{ORIP4e}
\label{ORIP4g}
\label{ORIP4d}
\label{ORIP5g}
\label{ORIP6a}
\label{ODEFIP}
\label{ORIP6f}
\label{ORIP7a}
\label{ORIP7g}

\psalmus{94}{VLrepet}

\rubric{Expleto Psalmo, dicitur Hymnus.}

\nocturn{1}
\rubric{Sub congruentibus Antiphonis dicuntur tres Psalmi,
ac deinde subjungitur Versus, prouti Officium occurrens requirit.
Versus cantatur hoc modo, et item respondetur:}

\gscore[n]{ORW}{T}{}{Versus}

\vspace{3\baselineskip}

\rubric{Post Versum cujuslibet Nocturni dicitur:}

\gscore[n]{ORPN}{T}{}{Pater Noster}

\vspace{3\baselineskip}

\smalltitle{Absolutio}
\gscore[n]{ORA}{T}{}{Absolutio}

\pagebreak

\smalltitle{Benedictiones et Lectiones}

\gscore[n]{ORLb}{T}{}{Benedictio!Tonus simplex}
\gscore[n]{ORLc}{T}{}{Benedictio!Tonus solemnis}

\rubric{Extra Chorum, quando ab uno tantum recitatur Officium,
ante singulas Lectiones, dicitur: \normaltext{Jube, Dómine, benedícere}
et subjungitur congruens Benedictio.
Ab Epsicopo autem, ultimam Matutini Lectionem cantaturo,
item dicitur: \normaltext{Jube, Dómine, benedícere};
et respondetur a Choro: \normaltext{Amen.}}

\rubric{Deinde dicuntur in unoquoque Nocturno Lectiones,
prouti Officium occurrens requirit, et in fine cujuslibet Lectionis additur:}

\gscore[n]{ORLd}{T}{}{In fine lectionum!Tonus simplex}
\gscore[n]{ORLe}{T}{}{In fine lectionum!Tonus solemnis}

\rubric{Post quamlibet vero Lectionem, quae Hymnum \normaltext{Te Deum}
immediate non præcedat, congruens dicitur Responsorium,
et in fine ultimi Responsorii cujuslibet Nocturni additur Versus:
\normaltext{Glória Patri, et Fílio, et Spirítui Sancto},
et Responsorium a signo \GreSpecial{+} repetitur, nisi aliter notatur.}

\rubric{Benedictiones pro aliis Lectionibus:}

\rubric{\emph{Benedictio 2.}} Unigénitus \textit{Dei} \textbf{Fí}lius~\GreSpecial{*}
nos benedícere et adjuváre dignétur.
\hspace{\specialcharhsep}\rr Amen.

\rubric{\emph{Benedictio 3.}} Spíritus \textit{Sancti} \textbf{grá}tia~\GreSpecial{*}
illúminet sensus et corda nostra.
\hspace{\specialcharhsep}\rr Amen.

\nocturn{2}

\rubric{Sub congruentibus item Antiphonis dicuntur tres Psalmi et Versus,
sicut in \Rnum{1} Nocturno.
Post Versus dicitur \normaltext{Pater noster} secreto usque ad
\vvrub \normaltext{Et ne nos indúcas in tentatiónem.}
\rrrub \normaltext{Sed líbera nos a malo.}}

\smalltitle{Absolutio}
\rubric{\emph{Absolutio 2.}}
Ipsíus píetas et misericódi\textit{a nos} \textbf{ád}juvet,~\GreSpecial{*}
qui cum Patre et Spíritu Sancto vivit et regnat in sǽcula sæculórum.
\hspace{\specialcharhsep}\rr Amen.

\smalltitle{Benedictiones}

\rubric{\emph{Benedictio 4.}} Deus Pa\textit{ter om}\textbf{ní}potens~\GreSpecial{*}
sit nobis propítius et clemens.
\hspace{\specialcharhsep}\rr Amen.

\rubric{\emph{Benedictio 5.}} Chris\textit{tus per}\textbf{pé}tuæ~\GreSpecial{*}
det nobis gaúdia vitæ.
\hspace{\specialcharhsep}\rr Amen.

\rubric{\emph{Benedictio 6.}} Ignem su\textit{i a}\textbf{mó}ris~\GreSpecial{*}
accéndat Deus in córdibus nostris.
\hspace{\specialcharhsep}\rr Amen.

\nocturn{3}

\rubric{Sub congruentibus item Antiphonis dicuntur tres Psalmi et Versus,
sicut in \Rnum{1} Nocturno.
Post Versus dicitur \normaltext{Pater noster} secreto usque ad
\vvrub \normaltext{Et ne nos indúcas in tentatiónem.}
\rrrub \normaltext{Sed líbera nos a malo.}}

\smalltitle{Absolutio}
\rubric{\emph{Absolutio 3.}}
A vínculis peccató\textit{rum nos}\textbf{tró}rum~\GreSpecial{*}
absólvat nos omnípotens et miséricors Dóminus.
\hspace{\specialcharhsep}\rr Amen.

\smalltitle{Benedictiones}

\rubric{\emph{Benedictio 7.}}
Evangé\textit{lica} \textbf{léc}tio~\GreSpecial{*}
sit nobis salus et protéctio.
\hspace{\specialcharhsep}\rr Amen.

\rubric{In Festis Domini et in Dominicis:}

\rubric{\emph{Benedictio 8.}}
Diví\textit{num au}\textbf{xí}lium~\GreSpecial{*}
máneat semper nobíscum.
\hspace{\specialcharhsep}\rr Amen.

\pagebreak
\rubric{In Festis beatæ Mariæ Viginis:}

\rubric{\emph{Benedictio 8.}}
Cujus \textit{festum} \textbf{có}limus,~\GreSpecial{*}
ipsa Virgo vírginum intercédat pro nobis ad Dóminum.
\hspace{\specialcharhsep}\rr Amen.

\rubric{In Festis Sanctorum:}

\rubric{\emph{Benedictio 8.}}
Cujus \rubric{(vel} Quarum\rubric{)} \textit{festum} \textbf{có}limus,~\GreSpecial{*}
ipse \rubric{(vel} ipsa \rubric{aut} ipsæ\rubric{)}
intercédat \rubric{(vel} intercédant\rubric{)} pro nobis ad Dóminum.
\hspace{\specialcharhsep}\rr Amen.

\rubric{\emph{Benedictio 9.}}
Ad societátem cívium \textit{super}\textbf{nó}rum~\GreSpecial{*}
perdúcat nos Rex Angelórum.
\hspace{\specialcharhsep}\rr Amen.

\rubric{Si autem legenda ultima Lectio sit de Homilia cum Evangelio Dominicæ,
vel Feriæ, aut Vigiliæ:}

\rubric{\emph{Benedictio 9.}}
Per evangé\textit{lica} \textbf{dic}ta~\GreSpecial{*}
deleántur nostra delícta.
\hspace{\specialcharhsep}\rr Amen.

\intermediatetitle{Te Deum}

\rubric{Post ultimam Lectionem, in omnibus Dominicis per Annum,
etiam repositis vel anticipatis, in Vigilia Epiphaniæ,
in Festis, excepto tamen sanctorum Innocentium Festo,
nisi hoc in Dominicam indicat, et per omnes Octavas, dicitur Hymnus Ambrosianus,
in tonus solemni pag.\ \pageref{M-ORTDa}, aut in tonus simplici pag.\ \pageref{M-ORTDb}.
In Adventu autem, et a Dominica Septuagesimæ usque ad Sabbatum sanctum inclusive,
non dicitur nisi in Festis. Quando vero Hymnus prædictus omittitur,
ejus loco dicitur \Rnum{9} Responsorium.}

\intermediatetitle{Conclusio}

\rubric{Dicto \normaltext{Te Deum}, aut ultimo Responsorio,
statim incipiuntur Laudes a Versu \normaltext{Deus, in adjutórium}.
Si non sequent Laudes, post Hymnum \normaltext{Te Deum},
vel post ultimum Responsorium, dicitur:}

\gscore[n]{ORDV}{T}{}{Dominus vobiscum}

\rubric{Hic versus non dicitur ab eo, qui non est saltem in ordine diaconatus;
sed ejus loco substituitur:}

\versiculus{Dómine, exáudi oratiónem meam.}{Et clamor meus ad te véniat.}
\vv Orémus.
\rubric{Deinde cantatur oratio et respondetur \normaltext{Amen.}}

\versiculus{Dóminus vobíscum.}{Et cum spíritu tuo.}
\rubric{vel \normaltext{Dómine, exáudi}, etc.}

\rubric{\normaltext{Benedicámus Dómino} secundum diem, pag.\ \pageref{M-TCBD}.}
\versiculus{Fidélium ánimæ per misericórdiam Dei requiéscant in pace.}{Amen.}

\rubric{Deinde dicitur \normaltext{Pater noster} totum secreto.}

\vspace{1cm}
\sep
\vspace{1cm}

\intermediatetitle{Post Divinum Officium}

\begin{multicols}{2}
\lettrine{S}{acrosánctæ} et indivíduæ Trinitáti, crucifíxi Dómini nostri Jesu Christi humanitáti, beatíssimæ et gloriosíssimæ sempérque Vírginis Maríæ fœcúndæ integritáti, et ómnium Sanctórum universitáti sit sempitérna laus, honor, virtus et glória ab omni creatúra, nobísque remíssio ómnium peccatórum, per infiníta sǽcula sæculórum. Amen.
\end{multicols}

\end{document}