% !TEX TS-program = lualatex
% !TEX encoding = UTF-8

\documentclass[psalterium-dominicis.tex]{subfiles}

\ifcsname preamble@file\endcsname
  \setcounter{page}{\getpagerefnumber{M-nr33_commune_mulierum}}
\fi

\begin{document}
\feast{MU}{Commune Virginum aut non Virginum}
	{Commune Sanctorum}{Commune Virginum aut non Virginum}{2}{}{}{}{}{}{}
\addcontentsline{toc}{section}{Commune Virginum aut non Virginum}

\rubric{Pro virgine:}
\gscore{MUVXIc}{I}{}{Regem Virginum (1.2.classis)}
\rubric{Pro virgine, Tempore Paschali:}
\gscore{MUVXId}{I}{}{Regem Virginum (TP)}
\rubric{Pro una non virgine:}
\gscore{MUNXIa}{I}{}{Laudemus (unius non virginis)}
\rubric{Pro plurimis non virginis:}
\gscore{MUNXIb}{I}{}{Laudemus (plurimarum non virginis)}
\rubric{Si Sancta fuerit Virgo tantum, et non Martyr, sive non Virgo, strophæ 1, 4 et 5 tantum dicuntur.}
\gscore{MUVMHa}{H}{}{Virginis proles}
\nocturn{1}
\gscore{MUXXN1A1}{A}{1}{O quam pulchra est casta}
\psalmus{8}{7}
\rubric{Pro virgine:}
\gscore{MUVXN1A2}{A}{2}{Ante torum hujus virginis}
\rubric{Pro non virgine:}
\gscore{MUNXN1A2}{A}{2}{Laeva ejus sub capite meo}
\psalmus{18}{4e}
\gscore{MUXXN1A3}{A}{3}{Revertere revertere sunamitis}
\psalmus{23}{7}
\versiculustpall{Spécie tua et pulchritúdine tua.}{Inténde, próspere procéde, et regna.}
\nocturn{2}
\gscore{MUXXN2A1}{A}{4}{Specie tua et pulchritudine tua intende}
\psalmus{44}{7}
\gscore{MUXXN2A2}{A}{5}{Adjuvabit eam}
\psalmus{45}{7}
\gscore{MUXXN2A3}{A}{6}{Aquae multae non potuerunt}
\psalmus{47}{8}
\versiculustpall{Adjuvábit eam Deus vultu suo.}{Deus in médio ejus, non commovébitur.}
\nocturn{3}
\gscore{MUXXN3A1}{A}{7}{Nigra sum}
\psalmus{95}{3}
\gscore{MUXXN3A2}{A}{8}{Trahe me post te in odorem}
\psalmus{96}{3}
\gscore{MUXXN3A3}{A}{9}{Veni sponsa Christi}
\psalmus{97}{8}
\versiculustpall{Elégit eam Deus, et prælégit eam.}{In tabernáculo suo habitáre facit eam.}
\end{document}