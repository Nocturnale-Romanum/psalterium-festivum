% !TEX TS-program = lualatex
% !TEX encoding = UTF-8

\documentclass[psalterium-dominicis.tex]{subfiles}

\ifcsname preamble@file\endcsname
  \setcounter{page}{\getpagerefnumber{M-pd43_sanctorale_aprilis}}
\fi

\begin{document}

\feast{0325}{In Annuntiatione Beatæ Mariæ Virginis}
	{Proprium Sanctorum}{Festa Martii}{1}{25 Martii}
	{}{}{Mariæ!Annuntiatio}
	{}
	{}
\rubric{Invitatorium ut infra, reliqua ut in Communi Festorum B.M.V. \pageref{M-CBMV}.}
\gscore{0325I}{I}{}{Ave Maria gratia plena}
\vspace{0.5\baselineskip}
\feast{Q5F6b}{Septem Dolorum Beatæ Mariæ Virginis}
	{Proprium Sanctorum}{Festa Martii}{2}{Feria VI post Dominicam Passionis}
	{}{}{Mariæ!Septem Dolorum}
	{}
	{}
\gscore{Q5F6I}{I}{}{Dolores gloriosae}
\pagebreak
\gscore{Q5F6H}{H}{}{Sancta Mater}
\nocturn{1}
\gscore{Q6F6N1A1}{A}{1}{Astiterunt reges terrae}
\psalmusref{2}{8}
\rubric{\aarub 2. \scorename{UMEXN1A3} pag.\ \pageref{M-UMEXN1A3} cum suo Psalmo \normaltext{3}.}
\gscore{0915N2A1}{A}{3}{Factum est cor meum}
\psalmus{12}{1}
\versiculus{Pósuit me desolátam.}{Tota die mæróre conféctam.}
\nocturn{2}
\gscore{Q5F6N2A1}{A}{4}{Inimici mei}
\pagebreak\null\par
\psalmus{40}{7}
\vspace{\baselineskip}
\rubric{\aarub 5. \scorename{0915N2A2} pag.\ \pageref{M-0915N2A2} cum suo Psalmo \normaltext{55}.}
\vspace{2\baselineskip}\null
\pagebreak
\gscore{Q5F6N2A3}{A}{6}{Filii hominum dentes eorum}
\psalmus{56}{2}
\versiculus{Fácies mea intúmuit a fletu.}{Et pálpebræ meæ caligavérunt.}
\pagebreak
\nocturn{3}
\rubric{\aarub 7. \scorename{0915N3A1} pag.\ \pageref{M-0915N3A1} cum suo Psalmo \normaltext{63}.}
\vspace{\baselineskip}
\gscore{Q5F6N3A2}{A}{8}{Factus sum sicut homo}
\psalmus{87}{4e}
\gscore{Q5F6N3A3}{A}{9}{Replevit me amaritudine}
\psalmus{108}{8}
\versiculus{Deus, vitam meam annuntiávi tibi.}{Posuísti lácrimas meas in conspéctu tuo.}
\pagebreak

\feast{0425}{S. Marci Evangelistæ}
	{Proprium Sanctorum}{Festa Aprilis}{2}{25 Aprilis}
	{}{}{Marci}
	{}
	{}
\rubric{Omnia de Communi Apostolorum tempore paschali, \pageref{M-APTP}.}

\feast{P2F4}{In Solemnitate Sancti Joseph\\Sponsi Beatæ Mariæ Virginis\\Confessoris et Ecclesiæ universalis Patroni}
	{Proprium Sanctorum}{Festa Aprilis}{2}{Feria IV infra Hebdomadam II post Pascha}
	{}{}{Joseph!Patronus}
	{}
	{}
\gscore{P2F4I}{I}{}{Laudemus... Joseph}
\gscore{P2F4H}{H}{}{Te Joseph}
\pagebreak
\nocturn{1}
\gscore{P2F4N1A}{A}{1}{Angelus Domini apparuit in somnis... fuge}
\psalmus{1}{3}
\psalmusref{2}{3}
\psalmusref{3}{3}
\versiculus{Confitébor nómini tuo, allelúia.}{Quóniam adjútor et protéctor factus es mihi, allelúia.}
\nocturn{2}
\gscore{P2F4N2A}{A}{2}{Angelus Domini apparuit in somnis... vade... defuncti}
\psalmus{4}{1}
\psalmus{5}{1}
\psalmusref{8}{1}
\versiculus{Réspice de cælo, et vide, et vísita víneam istam, allelúia.}{Et pérfice eam, allelúia.}
\pagebreak
\nocturn{3}
\gscore{P2F4N3A}{A}{3}{Consurgens Joseph... venit}
\psalmusref{14}{4e}
\psalmus{20}{4e}
\psalmusref{23}{4e}
\versiculus{Invocávi Dóminum, Patrem Dómini mei, allelúia.}{Ut non derelínquat me in die tribulatiónis, allelúia.}

\feast{0501}{Ss. Philippi et Jacobi Apostolorum}
	{Proprium Sanctorum}{Festa Maii}{2}{1 Maii (rubricæ Pii X) aut 11 Maii (rubricæ Joannis XXIII)}
	{}{}{Philippi et Jacobi}
	{}
	{}
\rubric{Omnia de Communi Apostolorum tempore paschali, \pageref{M-APTP}.}

\feast{0501b}{In Festo Sancti Joseph Opificis}
	{Proprium Sanctorum}{Festa Maii}{2}{1 Maii (rubricæ Joannis XXIII)}
	{}{}{Joseph!Opifex}
	{}
	{}
\gscore{0501I}{I}{}{Regem regum Dominum qui putari}
\gscore{0501H}{H}{}{Te pater Joseph opifex}
\pagebreak
\nocturn{1}
\gscore{0501N1A}{A}{1}{Exit homo ad opus}
\psalmus{1}{1}
\psalmus{2}{1}
\vspace{\baselineskip}
\psalmus{3}{1}
\versiculus{Glória et exémplar opíficum, sancte Joseph, allelúia.}{Cui obedíre vóluit Fílius Dei, allelúia.}
\nocturn{2}
\gscore{0501N2A}{A}{2}{Jesus cum esset triginta}
\psalmus{4}{8}
\vspace{\baselineskip}
\psalmus{5}{8}
\vspace{\baselineskip}
\psalmus{8}{8}
\versiculus{O magnam diginátem laboris, allelúia.}{Quem Christus sanctificávit, allelúia.}
\vspace{2\baselineskip}
\nocturn{3}
\gscore{0501N3A}{A}{3}{Nonne hic est fabri filius}
\vspace{\baselineskip}
\psalmus{14}{7}
\pagebreak
\psalmus{20}{7}
\vspace{\baselineskip}
\psalmus{23}{7}
\versiculus{Verbum Dei, per quod facta sunt ómnia, allelúia.}{Dignátus est operári mánibus suis, allelúia.}

\feast{0503}{In Inventione Sanctæ Crucis}
	{Proprium Sanctorum}{Festa Maii}{2}{3 Maii}
	{}{}{Jesu Christi, Domini nostri!Inventio Crucis}
	{}
	{}
\vspace{-0.5\baselineskip}
\gscore{0503I}{I}{}{Christum Regem crucifixum}
\rubric{Hymnus ut in Dominicæ Passionis, pag.\ \pageref{M-Q5H}.}
\nocturn{1}
\vspace{-0.5\baselineskip}
\gscore{0503N1A}{A}{1}{Inventae Crucis}
\pagebreak
\psalmusref{1}{1}
\psalmusref{2}{1}
\psalmusref{3}{1}
\versiculus{Hoc signum Crucis erit in cælo, allelúia.}{Cum Dóminus ad judicándum vénerit, allelúia.}
\nocturn{2}
\gscore{0503N2A}{A}{2}{Felix ille triumphus}
\psalmus{4}{2}
\pagebreak
\psalmus{5}{2}
\psalmus{8}{2}
\versiculus{Adorámus te, Christe, et benedícimus tibi, allelúia.}{Quia per Crucem tuam redemísti mundum, allelúia.}

\nocturn{3}
\gscore{0503N3A}{A}{3}{Adoramus te \emph{cum} Alleluia}
\psalmusref{95}{1}
\psalmusref{96}{1}
\psalmus{97}{1}
\versiculus{Omnis terra adóret te, et psallat tibi, allelúia.}{Psalmum dicat nómini tuo, allelúia.}


\feast{0506}{S. Joannis Apostoli ante Portam Latinam}
	{Proprium Sanctorum}{Festa Maii}{2}{6 Maii}
	{}{}{Joannis Evangelistæ!Ante Portam Latinam}
	{}
	{}
\rubric{Omnia de Communi Apostolorum Tempore Paschali, \pageref{M-APTP}.}

\feast{0508}{In Apparitione S. Michaëlis Archangeli}
	{Proprium Sanctorum}{Festa Maii}{2}{8 Maii}
	{}{}{Michaëlis!Apparitio}
	{}
	{}
\rubric{Invitatorium \scorename{0324I} et Hymnus \scorename{0508H} pag.\ \pageref{M-0508H}.}
\nocturn{1}
\gscore{0508N1A}{A}{1}{Concussum \emph{cum} Alleluia}
\psalmusref{8}{8}
\psalmusref{10}{8}
\pagebreak
\psalmus{14}{8}
\versiculus{Stetit Angelus juxta aram templi, allelúia.}{Habens thuríbulum áureum in manu sua, allelúia.}
\nocturn{2}
\gscore{0508N2A}{A}{2}{Michael Archangele \emph{cum} Alleluia}
\psalmusref{18}{2}
\psalmusref{23}{2}
\psalmus{33}{2}
\versiculus{Ascéndit fumus aromátum in conspéctu Dómini, allelúia.}{De manu Angeli, allelúia.}
\nocturn{3}
\gscore{0929N3A1}{A}{3}{Angelus Archangelus}
\psalmusref{95}{7}
\psalmusref{96}{7}
\psalmus{102}{7}
\vspace{2\baselineskip}
\versiculus{In conspéctu Angelórum psallam tibi, Deus meus, allelúia.}{Adorábo ad templum sanctum tuum, et confitébor nómini tuo, allelúia.}
\vfill\pagebreak

\feast{0531b}{In Festo Beatæ Mariæ Virginis Reginæ}
	{Proprium Sanctorum}{Festa Maii}{2}{31 Maii}
	{}{}{Mariæ!Regina}
	{}
	{}
\rubric{Omnia de Communi B.M.V., pag.\pageref{M-CBMV}, præter sequentes.}
\gscore{0531I}{I}{}{Christum Regem qui suam}
\gscore{0531H}{H}{}{Rerum supremo}
\nocturn{1}
\versiculus{Salve, Regina misericórdiæ, allelúia.}{Ex qua natus est Christus, Rex noster, allelúia.}
\vspace{\baselineskip}
\nocturn{2}
\versiculus{Stabat juxta crucem Jesu Mater ejus, allelúia.}{In passióne sócia, totíus mundi Regina, allelúia.}
\vspace{\baselineskip}
\nocturn{3}
\versiculus{Beátam me dicent omnes generatiónes, allelúia.}{Quia fecit mihi magna qui potens est, allelúia.}

\end{document}