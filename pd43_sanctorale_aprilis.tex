% !TEX TS-program = lualatex
% !TEX encoding = UTF-8

\documentclass[psalterium-dominicis.tex]{subfiles}

\ifcsname preamble@file\endcsname
  \setcounter{page}{\getpagerefnumber{M-pd43_sanctorale_aprilis}}
\fi

\begin{document}

\feast{0325}{In Annuntiatione Beatæ Mariæ Virginis}
	{Proprium Sanctorum}{Festa Martii}{1}{25 Martii}
	{}{}{Mariæ!Annuntiatio}
	{}
	{}
\rubric{Invitatorium ut infra, reliqua ut in Communi Festorum B.M.V. \pageref{M-CBMV}.}
\gscore{0325I}{I}{}{Ave Maria gratia plena}

\feast{Q5F6b}{Septem Dolorum Beatæ Mariæ Virginis}
	{Proprium Sanctorum}{Festa Martii}{2}{Feria VI post Dominicam Passionis}
	{}{}{Mariæ!Septem Dolorum}
	{}
	{}
\gscore{Q5F6I}{I}{}{Dolores gloriosae recolentes}
\gscore{Q5F6H}{H}{}{Sancta Mater istud agas}
\nocturn{1}
\gscore{Q5F6N1A1}{A}{1}{Astiterunt reges terrae}
\psalmus{2}{8}
\gscore{Q5F6N1A2}{A}{2}{Voce mea ad Dominum}
\psalmus{3}{7}
\gscore{Q5F6N1A3}{A}{3}{Factum est cor meum tamquam cera}
\psalmus{12}{}
\versiculus{Pósuit me desolátam.}{Tota die mœróre conféctam.}
\nocturn{2}
\gscore{Q5F6N2A1}{A}{4}{Inimici mei dixerunt mala mihi quando}
\psalmus{40}{7}
\gscore{Q5F6N2A2}{A}{5}{Deus vitam meam annuntiavi tibi}
\psalmus{55}{7}
\gscore{Q5F6N2A3}{A}{6}{Filii hominum dentes eorum}
\psalmus{56}{2}
\versiculus{Fácies mea intúmuit a fletu.}{Et pálpebræ meæ caligavérunt.}
\nocturn{3}
\gscore{Q5F6N3A1}{A}{7}{Intenderunt arcum}
\psalmus{63}{8}
\gscore{Q5F6N3A2}{A}{8}{Factus sum sicut homo}
\psalmus{87}{1}
\gscore{Q5F6N3A3}{A}{9}{Replevit me a maritudine}
\psalmus{108}{8}
\versiculus{Deus, vitam meam annuntiávi tibi.}{Posuísti lácrimas meas in conspéctu tuo.}

\feast{0425}{S. Marci Evangelistæ}
	{Proprium Sanctorum}{Festa Aprilis}{2}{25 Aprilis}
	{}{}{Marci}
	{}
	{}
\rubric{Omnia de Communi Apostolorum tempore paschali, \pageref{M-APTP}.}

\feast{P2F4}{In Solemnitate Sancti Joseph\\Sponsi Beatæ Mariæ Virginis\\Confessoris et Ecclesiæ universalis Patroni}
	{Proprium Sanctorum}{Festa Aprilis}{2}{Feria IV infra Hebdomadam II post Pascha}
	{}{}{Joseph!Patronus}
	{}
	{}
\rubric{Psalmi trium Nocturnorum de Communi Confessoris non Pontificis \pageref{M-CONP}, reliqua ut infra notatur.}
\gscore{P2F4I}{I}{}{Laudemus... Joseph}
\gscore{P2F4H}{H}{}{Te Joseph celebrent agmina}
\nocturn{1}
\gscore{P2F4N1A}{A}{1}{Angelus Domini apparuit in somnis... fuge}
\versiculus{Confitébor nómini tuo, allelúja.}{Quóniam adjútor et protéctor factus es mihi, allelúja.}
\nocturn{2}
\gscore{P2F4N2A}{A}{2}{Angelus Domini apparuit in somnis... vade}
\versiculus{Réspice de cœlo, et vide, et vísita víneam istam, allelúja.}{Et pérfice eam, allelúja.}
\nocturn{3}
\gscore{P2F4N3A}{A}{3}{Consurgens Joseph... venit}
\versiculus{Invocávi Dóminum, Patrem Dómini mei, allelúja.}{Ut non derelínquat me in die tribulatiónis, allelúja.}


\feast{0501}{Ss. Philippi et Jacobi Apostolorum}
	{Proprium Sanctorum}{Festa Maii}{2}{1 Maii (rubricæ Pii X) aut 11 Maii (rubricæ Joannis XXIII)}
	{}{}{Philippi et Jacobi}
	{Omnia de Communi Apostolorum tempore paschali, \pageref{M-APTP}.}
	{}

\feast{0501b}{In Festo Sancti Joseph Opificis}
	{Proprium Sanctorum}{Festa Maii}{2}{1 Maii (rubricæ Joannis XXIII)}
	{}{}{Joseph!Opifex}
	{}
	{Psalmi trium nocturnorum de Communi Confessoris non Pontificis \pageref{M-CONP}, reliqua ut infra notantur.}
\gscore{0501I}{I}{}{Regem regum Dominum qui putari}
\gscore{0501H}{H}{}{Te pater Joseph opifex}
\nocturn{1}
\gscore{0501N1A}{A}{1}{Exit homo ad opus}
\versiculus{Glória et exémplar opíficum, sancte Joseph, allelúja.}{Cui obœdíre vóluit Fílius Dei, allelúja.}
\nocturn{2}
\gscore{0501N2A}{A}{2}{Jesus cum esset triginta}
\versiculus{O magnam diginátem laboris, allelúja.}{Quem Christus sanctificávit, allelúja.}
\nocturn{3}
\gscore{0501N3A}{A}{3}{Nonne hic est fabri filius}
\versiculus{Verbum Dei, per quod facta sunt ómnia, allelúja.}{Dignátus est operári mánibus suis, allelúja.}

\feast{0503}{In Inventione Sanctæ Crucis}
	{Proprium Sanctorum}{Festa Maii}{2}{3 Maii}
	{}{}{Jesu Christi, Domini nostri!Inventio Crucis}
	{}
	{}
\gscore{0503I}{I}{}{Christum Regem crucifixum venite}
\gscore{Q5H}{H}{}{Pange lingua... Lauream}
\nocturn{1}
\rubric{Tres Psalmi ut in \Rnum{1} Nocturno Communis unius Martyris extra tempus paschale, pag.\ \pageref{M-UMEXN1A1} sub hac sola Antiphona cantatur.}
\gscore{0503N1A}{A}{1}{Inventae Crucis festa recolimus}
\versiculus{Hoc signum Crucis erit in cœlo, allelúja.}{Cum Dóminus ad judicándum vénerit, allelúja.}
\nocturn{2}
\rubric{Tres Psalmi ut in \Rnum{2} Nocturno Communis unius Martyris extra tempus paschale, pag.\ \pageref{M-UMEXN2A1} sub hac sola Antiphona cantatur.}
\gscore{0503N2A}{A}{2}{Felix ille triumphus}
\versiculus{Adorámus te, Christe, et benedícimus tibi, allelúja.}{Quia per Crucem tuam redemísti mundum, allelúja.}
\nocturn{3}
\rubric{Tres Psalmi ut in \Rnum{3} Nocturno Communis Virginum, pag.\ \pageref{M-MUXXN3A1} sub hac sola Antiphona cantatur.}
\gscore{0503N3A}{A}{3}{Adoramus te Christe... quia per Crucem}
\versiculus{Omnis terra adóret te, et psallat tibi, allelúja.}{Psalmum dicat nómini tuo, allelúja.}

\feast{0506}{S. Joannis Apostoli ante Portam Latinam}
	{Proprium Sanctorum}{Festa Maii}{2}{6 Maii}
	{}{}{Joannis Evangelistæ!Ante Portam Latinam}
	{}
	{}
\rubric{Omnia de Communi Apostolorum Tempore Paschali, \pageref{M-APTP}.}

\feast{0508}{In Apparitione S. Michaëlis Archangeli}
	{Proprium Sanctorum}{Festa Maii}{2}{8 Maii}
	{}{}{Michaëlis!Apparitio}
	{Invitatorium \scorename{0324I} pag.\ \pageref{M-0324I}, psalmi trium Nocturnorum ut in Festo Gabrielis Archangeli, pag.\ \pageref{M-0324}, reliqua ut infra.}
	{}
\gscore{0508H}{H}{}{Te splendor et virtus Patris}
\nocturn{1}
\gscore{0508N1A}{A}{1}{Concussum est mare... alleluia}
\versiculus{Stetit Angelus juxta aram templi, allelúja.}{Habens thuríbulum áureum in manu sua, allelúja.}
\nocturn{2}
\gscore{0508N2A}{A}{2}{Michael Archangele venit in adjutorium... alleluia}
\versiculus{Ascéndit fumus aromátum in conspéctu Dómini, allelúja.}{De manu Angeli, allelúja.}
\nocturn{3}
\gscore{0508N3A}{A}{3}{Angelus Archangelus Michael Dei nuntius... alleluia alleluia}
\versiculus{In conspéctu Angelórum psallam tibi, Deus meus, allelúja.}{Adorábo ad templum sanctum tuum, et confitébor nómini tuo, allelúja.}

\feast{0531b}{In Festo Beatæ Mariæ Virginis Reginæ}
	{Proprium Sanctorum}{Festa Maii}{2}{31 Maii (rubricæ Joannis XXIII)}
	{}{}{Mariæ!Reginæ}
	{}
	{Omnia ut in Communi B.M.V., pag.\pageref{M-CBMV}, præter ea quæ hic habentur propria.}
\gscore{0531I}{I}{}{Christum regem qui suam coronavit Matrem}
\gscore{0531H}{H}{}{Rerum suprem in vertice}
\nocturn{1}
\versiculus{Salve, Regina misericórdiæ, allelúja.}{Ex qua natus est Christus, Rex noster, allelúja.}
\nocturn{2}
\versiculus{Stabat juxta crucem Jesu Mater ejus, allelúja.}{In passióne sócia, totíus mundi Regina, allelúja.}
\nocturn{3}
\versiculus{Beátam me dicent omnes generatiónes, allelúja.}{Quia fecit mihi magna qui potens est, allelúja.}

\end{document}