% !TEX TS-program = lualatex
% !TEX encoding = UTF-8

\documentclass[psalterium-festivum.tex]{subfiles}

\ifcsname preamble@file\endcsname
  \setcounter{page}{\getpagerefnumber{M-pd60_variationes}}
\fi

\begin{document}

\addcontentsline{toc}{chapter}{Variationes juxta rubricas Joannis \Rnum{23}}

\feast{OR}{Variationes in Divino Officio\\juxta rubricas Joannis \textsc{pp} \Rnum{23}}
	{Variationes}{Variationes}{2}{}{}{}{}{}{}
\thispagestyle{empty}

\titlerubric{De Festis \Rnum{3} classis}

~

\rubric{In Festis \Rnum{3} classis dicuntur Psalmi et Versus de Feriæ ut in Festis trium lectionum, aut Psalmi proprii, juxta rubricas. 
Si dicuntur Psalmi proprii, dicitur singulus Versus ut datur in tertio Nocturno Festi.}

~

\titlerubric{De Kalendario}

~

\rubric{Omittuntur vel transferuntur Festa, secundum Kalendarium.
De Octavis S. Joannis Baptistæ, SS. Petri et Pauli, Assumptionis B.M.V., Omnium Sanctorum, et Immaculatæ Conceptionis B.M.V., nihil fit.}

~

\titlerubric{De Hymno \emph{Iste Confessor}}

~

\rubric{In Hymno \emph{Iste Confessor} semper mutatur tertius versus.}

~

\titlerubric{De Psalmo 88}

~

\rubric{In Festo Nativitatis Domini, Palmus 88 dicitur usque ad \normaltext{et testis in cælo fidélis}, et additur \normaltext{Glória Patri}.}

\rubric{In Festo Transfigurationis Domini, Psalmus 88 dicitur usque ad \normaltext{et Sancti Israël, regis nostri}, et additur \normaltext{Glória Patri}.}


\end{document}
