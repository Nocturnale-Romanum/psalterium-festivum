% !TEX TS-program = lualatex
% !TEX encoding = UTF-8

\documentclass[psalterium-dominicis.tex]{subfiles}

\ifcsname preamble@file\endcsname
  \setcounter{page}{\getpagerefnumber{M-pd60_variationes}}
\fi

\begin{document}

\addcontentsline{toc}{chapter}{Variationes juxta rubricas Joannis \Rnum{23}}

\feast{OR}{Variationes in Divino Officio\\juxta rubricas Joannis \textsc{pp} \Rnum{23}}
	{Variationes}{Variationes}{2}{}{}{}{}{}{}
\thispagestyle{empty}

\intermediatetitle{Proprium de Tempore}

\rubric{In Dominicis Temporis Adventus, dicitur singulus Versus:}

\versiculus{Ex Sion spécies decóris ejus.}{Deus noster maniféste véniet.}

\rubric{In Dominicis per annum, dicitur singulus Versus:}

\versiculus{Prævenérunt óculi mei ad te dilúculo.}{Ut meditárer elóquia tua, Dómine.}

\rubric{In Dominicis Temporis Quadragesimæ, dicitur singulus Versus:}

\versiculus{Ipse liberávit me de láqueo venántium.}{Et a verbo áspero.}

\rubric{In Dominicis Temporis Passionis, dicitur singulus Versus:}

\versiculus{Erue a frámea, Deus, ánimam meam.}{Et de manu canis únicam meam.}

\rubric{In Dominicis Temporis Paschalis, dicuntur novem Psalmi sub singula antiphona \scorename{P1F1N1A}, et singulus Versus:}

\versiculus{Surréxit Dóminus de sepúlcro, allelúia.}{Qui pro nobis pepéndit in ligno, allelúia.}

\rubric{In Dominica post Ascensionem, dicuntur novem Psalmi de Dominica sub singula Antiphona \normaltext{Allelúia, allelúia, allelúia}, et singulus Versus:}

\versiculus{Ascéndit Deus in jubilatióne, allelúia.}{Et Dóminus in voce tubæ, allelúia.}

\intermediatetitle{Proprium Sanctorum}

\rubric{In Festis \Rnum{3} classis dicuntur Psalmi et Versus de Feriæ, aut Psalmi proprii, juxta rubricas. 
Si dicuntur Psalmi proprii, dicitur singulus Versus ut datur in tertio Nocturno Festi.
Omittuntur vel transferuntur Festa, secundum Kalendarium.}

\end{document}
