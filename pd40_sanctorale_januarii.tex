% !TEX TS-program = lualatex
% !TEX encoding = UTF-8

\documentclass[psalterium-dominicis.tex]{subfiles}

\ifcsname preamble@file\endcsname
  \setcounter{page}{\getpagerefnumber{M-pd40_sanctorale_januarii}}
\fi

\begin{document}
\feast{PRSA}{Proprium Sanctorum}{Proprium Sanctorum}{Proprium Sanctorum}{1}{}{}{}{}{}{}
\addcontentsline{toc}{chapter}{Proprium Sanctorum}

\feast{0118}{In Cathedra Sancti Petri Apostoli Romæ}
	{Proprium Sanctorum}{Festa Januarii}{2}{18 Januarii}
	{}{}{Petri Apostoli!Cathedra Romæ}
	{}
	{}
\rubric{Omnia ut in Communi Confessoris non Pontificis, \pageref{M-CONP}, præter sequentes.}
\gscore{0118I}{I}{}{Tu es pastor ovium}
\gscore{0118H}{H}{}{Quodcumque vinclis}
\nocturn{2}
\versiculus{Elégit eum Dóminus sacerdótem sibi.}{Ad sacrificándum ei hóstiam laudis.}
\nocturn{3}
\versiculus{Tu es sacérdos in ætérnum.}{Secúndum órdinem Melchísedech.}

\feast{0125}{In Conversione Sancti Pauli Apostoli}
	{Proprium Sanctorum}{Festa Januarii}{2}{25 Januarii}
	{}{}{Pauli Apostoli!Conversio}
	{}
	{}
\gscore{0125I}{I}{}{Laudemus... Doctoris gentium}
\gscore{0125H}{H}{}{Egregie Doctor Paule}
\nocturn{1}
\gscore{0125N1A1}{A}{1}{Qui operatus est Petro}
\psalmus{18}{1}
\gscore{0125N1A2}{A}{2}{Scio cui credidi}
\psalmus{33}{1}
\gscore{0125N1A3}{A}{3}{Mihi vivere Christus est}
\psalmus{44}{1}
\versiculus{In omnem terram exívit sonus eórum.}{Et in fines orbis terræ verba eórum.}
\nocturn{2}
\gscore{0125N2A1}{A}{4}{Tu es vas electionis}
\psalmus{46}{8}
\gscore{0125N2A2}{A}{5}{Magnus sanctus Paulus}
\psalmus{60}{8}
\gscore{0125N2A3}{A}{6}{Bonum certamen certavi}
\psalmus{63}{2}
\versiculus{Constítues eos príncipes super omnem terram.}{Mémores erunt nóminis tui, Dómine.}
\nocturn{3}
\gscore{0125N3A1}{A}{7}{Saulus qui et Paulus magnus}
\psalmus{74}{8}
\gscore{0125N3A2}{A}{8}{Ne magnitudo revelationum}
\psalmus{96}{1}
\gscore{0125N3A3}{A}{9}{Reposita est mihi corona}
\psalmus{98}{8}
\versiculus{Nimis honoráti sunt amíci tui, Deus.}{Nimis confortátus est principátus eórum.}

\feast{0202}{In Purificatione Beatæ Mariæ Virginis}
	{Proprium Sanctorum}{Festa Februarii}{2}{2 Februarii}
	{}{}{Mariæ!Purificatio}
	{}
	{}
\rubric{Omnia de Communi Festorum B.M.V.\ pag.\ \pageref{M-CBMV}, præter sequens.}
\gscore{0202I}{I}{}{Ecce venit ad templum}

\feast{0211}{In Apparitione Beatæ Mariæ Virginis Immaculatæ}
	{Proprium Sanctorum}{Festa Februarii}{2}{11 Februarii}
	{}{}{Mariæ!Apparitio Immaculatæ}
	{Psalmi de Communi Festorum B.M.V. \pageref{M-CBMV}, reliqua ut infra.}
	{}
\gscore{1208I}{I}{}{Immaculatam Conceptionem}
\gscore{0211H}{H}{}{Te dicimus praeconio}
\nocturn{1}
\gscore{0211N1A1}{A}{1}{Ave gratia plena dominus tecum}
\psalmus{2}{}
\gscore{0211N1A2}{A}{2}{Benedicta tu inter mulieres}
\psalmus{18}{}
\gscore{0211N1A3}{A}{3}{Ne timeas Maria}
\psalmus{23}{}
\versiculus{Deus omnípotens præcínxit me virtúte.}{Et pósuit immaculátam viam meam.}
\nocturn{2}
\gscore{0211N2A1}{A}{4}{Fecit mihi magna qui potens est}
\psalmus{44}{}
\gscore{0211N2A2}{A}{5}{Sanctificavit tabernacula}
\psalmus{45}{}
\gscore{0211N2A3}{A}{6}{Dominus possedit te}
\psalmus{86}{}
\versiculus{Adjuvábit eam Deus vultu suo.}{Deus in médio ejus, non commovébitur.}
\nocturn{3}
\gscore{0211N3A1}{A}{7}{Manus Domini confortavit}
\psalmus{95}{}
\gscore{0211N3A2}{A}{8}{Noli metuere}
\psalmus{96}{}
\gscore{0211N3A3}{A}{9}{Benedixit te Dominus in virtute}
\psalmus{97}{}
\versiculus{Diffúsa est grátia in lábiis tuis.}{Proptérea benedíxit te Deus in ætérnum.}

\feast{0222}{In Cathedra S. Petri Apostoli Antiochiæ}
	{Proprium Sanctorum}{Festa Februarii}{2}{22 Februarii}
	{}{}{Petri Apostoli!Cathedra Antiochiæ}
	{}
	{}
\rubric{Omnia ut in Cathedra S.\ Petri Apostoli Romæ, pag.\ \pageref{M-0118}.}

\feast{0224}{S. Matthiæ Apostoli}
	{Proprium Sanctorum}{Festa Februarii}{2}{24 Februarii aut 25 in anno bissextili}
	{}{}{Matthiæ}
	{}
	{}
\rubric{Omnia de Communi Apostolorum, \pageref{M-APEX}.}

\end{document}
