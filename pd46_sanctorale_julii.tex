% !TEX TS-program = lualatex
% !TEX encoding = UTF-8

\documentclass[psalterium-dominicis.tex]{subfiles}

\ifcsname preamble@file\endcsname
  \setcounter{page}{\getpagerefnumber{M-pd46_sanctorale_julii}}
\fi

\begin{document}

\feast{0701}{Pretiosissimi Sanguinis\\Domini nostri Jesu Christi}
	{Proprium Sanctorum}{Festa Julii}{2}{1 Julii}
	{}{}{Jesu Christi, Domini nostri!Sanguis}
	{}
	{}
\gscore{0701I}{I}{}{Christum Dei Filium qui suo nos redemit}
\gscore{0701H}{H}{}{Ira justa Conditoris}
\nocturn{1}
\gscore{0701N1A1}{A}{1}{Postquam consummati sunt}
\psalmus{2}{1}
\gscore{0701N1A2}{A}{2}{Factus in agonia prolixius orabat}
\psalmus{3}{3}
\gscore{0701N1A3}{A}{3}{Judas qui eum tradidit poenitentia ductus}
\psalmus{15}{8}
\versiculus{Redemísti nos, Dómine.}{In sánguine tuo.}
\nocturn{2}
\gscore{0701N2A1}{A}{4}{Pilatus volens populo satis}
\psalmus{22}{1}
\gscore{0701N2A2}{A}{5}{Videns autem quia nihil proficeret}
\psalmus{29}{3}
\gscore{0701N2A3}{A}{6}{Et respondens universus}
\psalmus{63}{8}
\versiculus{Sanguis Jesu Christi, Fílii Dei.}{Emúndat nos ab omni peccáto.}
\nocturn{3}
\gscore{0701N3A1}{A}{7}{Exivit ergo Jesus portans coronam}
\psalmus{73}{6}
\gscore{0701N3A2}{A}{8}{Et bajulans sibi crucem}
\psalmus{87}{5}
\gscore{0701N3A3}{A}{9}{Ut viderunt eum jam mortuum}
\psalmus{93}{1}
\versiculus{Christus diléxit nos.}{Et lavit nos a peccátis nostris in sánguine suo.}

\feast{0702}{In Visitatione Beatæ Mariæ Virginis}
	{Proprium Sanctorum}{Festa Julii}{2}{2 Julii}
	{}{}{Mariæ!Visitatio}
	{}
	{}
\rubric{Omnia de Communi B.M.V. pag. \pageref{M-CBMV}, præter sequens.}
\gscore{0702I}{I}{}{Visitationem}

\feast{0725}{S. Jacobi Apostoli}
	{Proprium Sanctorum}{Festa Julii}{2}{25 Julii}
	{}{}{Jacobi Apostoli}
	{}
	{}
\rubric{Omnia de Communi Apostolorum, pag.\ \pageref{M-APEX}.}

\feast{0726}{S. Annæ Matris B. M. V.}
	{Proprium Sanctorum}{Festa Julii}{2}{26 Julii}
	{}{}{Annæ}
	{}
	{}
\rubric{Omnia de Communi Virginum, pag.\ \pageref{M-MU}.}
\gscore{MUNXIa}{I}{}{Laudemus... Annae}
\rubric{Si Sancta fuerit Virgo tantum, et non Martyr, sive non Virgo, strophæ 1, 4 et 5 tantum dicuntur.}
\gscore{MUVMHa}{H}{}{Virginis Proles}
\nocturn{1}
\gscore{MUXXN1A1}{A}{1}{O quam pulchra est casta}
\psalmus{8}{7}
\gscore{MUNXN1A2}{A}{2}{Laeva ejus sub capite meo}
\psalmus{18}{4e}
\gscore{MUXXN1A3}{A}{3}{Revertere revertere sunamitis}
\psalmus{23}{7}
\versiculus{Spécie tua et pulchritúdine tua.}{Inténde, próspere procéde, et regna.}
\nocturn{2}
\gscore{MUXXN2A1}{A}{4}{Specie tua et pulchritudine tua intende}
\psalmus{44}{7}
\gscore{MUXXN2A2}{A}{5}{Adjuvabit eam}
\psalmus{45}{7}
\gscore{MUXXN2A3}{A}{6}{Aquae multae non potuerunt}
\psalmus{47}{8}
\versiculus{Adjuvábit eam Deus vultu suo.}{Deus in médio ejus, non commovébitur.}
\nocturn{3}
\gscore{MUXXN3A1}{A}{7}{Nigra sum}
\psalmus{95}{3}
\gscore{MUXXN3A2}{A}{8}{Trahe me post te in odorem}
\psalmus{96}{3}
\gscore{MUXXN3A3}{A}{9}{Veni sponsa Christi}
\psalmus{97}{8}
\versiculus{Elégit eam Deus, et præelégit eam.}{In tabernáculo suo habitáre facit eam.}


\feast{0801}{S. Petri ad Vincula}
	{Proprium Sanctorum}{Festa Augusti}{2}{1 Augusti}
	{Duplex majus}{(Omittitur)}{Petri Apostoli!ad Vincula}
	{}
	{}
\rubric{Omnia de Communi Apostolorum pag.\ \pageref{M-APEX}, præter Hymnus \scorename{0118H}, pag.\ \pageref{M-0118H}.}

\feast{0805}{In Dedicatione Sanctæ Mariæ ad Nives}
	{Proprium Sanctorum}{Festa Augusti}{2}{5 Augusti}
	{}{}{Mariæ!Dedicatio S. M. ad Nives}
	{}
	{}
\rubric{Omnia de Communi B.M.V., pag.\ \pageref{M-CBMV}.}

\feast{0806}{In Transfiguratione\\Domini nostri Jesu Christi}
	{Proprium Sanctorum}{Festa Augusti}{2}{6 Augusti}
	{}{}{Jesu Christi, Domini nostri!Transfiguratio}
	{}
	{}
\gscore{0806I}{I}{}{Summum Regem}
\gscore{0806H}{H}{}{Quicumque Christum}
\nocturn{1}
\gscore{0806N1A1}{A}{1}{Paulo minus ab Angelis minoratus}
\psalmus{8}{1}
\gscore{0806N1A2}{A}{2}{Revelavit Dominus condensa}
\psalmus{28}{1}
\gscore{0806N1A3}{A}{3}{Speciosus forma prae filis hominum}
\psalmus{44}{1}
\versiculus{Gloriósus apparuísti in conspéctu Dómini.}{Proptérea decórem induit te Dóminus.}
\nocturn{2}
\gscore{0806N2A1}{A}{4}{Illuminans tu mirabiliter}
\psalmus{75}{4e}
\gscore{0806N2A2}{A}{5}{Melior est dies una in atriis tuis}
\psalmus{83}{5}
\gscore{0806N2A3}{A}{6}{Gloriosa dicta sunt de te civitas Dei}
\psalmus{86}{6}
\versiculus{Glória et honóre coronásti eum, Dómine.}{Et Constituísti eum super ópera mánuum tuárum.}
\nocturn{3}
\gscore{0806N3A1}{A}{7}{Thabor et Hermon in nomine tuo}
\psalmus{88}{7}
\gscore{0806N3A2}{A}{8}{Lux orta est justo et rectis corde laetitia}
\psalmus{96}{8}
\gscore{0806N3A3}{A}{9}{Confessionem et decorem induit}
\psalmus{103}{1}
\versiculus{Magna est glória ejus in salutári tuo.}{Glóriam et magnum decórem impónes super eum.}

\feast{0810}{S. Laurentii Martyris}
	{Proprium Sanctorum}{Festa Augusti}{2}{10 Augusti}
	{}{}{Laurentii}
	{}
	{}
\gscore{0810I}{I}{}{Beatus Laurentius}
\rubric{Hymnus \scorename{UMEXHa}, pag.\ \pageref{M-UMEXHa}.}
\nocturn{1}
\gscore{0810N1A1}{A}{1}{Quo progrederis sine filio}
\psalmus{1}{7}
\gscore{0810N1A2}{A}{2}{Non me derelinque pater sancte}
\psalmus{2}{8}
\gscore{0810N1A3}{A}{3}{Non ego te desero fili}
\psalmus{3}{8}
\versiculus{Glória et honóre coronásti eum, Dómine.}{Et constituísti eum super ópera mánuum tuárum.}
\nocturn{2}
\gscore{0810N2A1}{A}{4}{Beatus Laurentius orabat}
\psalmus{4}{8}
\gscore{0810N2A2}{A}{5}{Dixit Romanus ad beatum Laurentium}
\psalmus{5}{7}
\gscore{0810N2A3}{A}{6}{Beatus Laurentius dixit}
\psalmus{8}{8}
\versiculus{Posuísti, Dómine, super caput ejus.}{Corónam de lápide pretióso.}
\nocturn{3}
\gscore{0810N3A1}{A}{7}{Strinxerunt corporis}
\psalmus{10}{7}
\gscore{0810N3A2}{A}{8}{Igne me examinasti}
\psalmus{16}{8}
\gscore{0810N3A3}{A}{9}{Interrogatus te Dominum confessus sum}
\psalmus{20}{7}
\versiculus{Magna est glória ejus in salutári tuo.}{Glóriam et magnum decórem impónes super eum.}

\end{document}